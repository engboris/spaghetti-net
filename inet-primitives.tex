\input{style}
\usepackage[utf8]{inputenc}
\usepackage{amssymb}
\usepackage{amsmath}
\usepackage{graphicx}
\usepackage{fancybox}
\usepackage{xifthen}
\usepackage{cmll}
\usepackage{stmaryrd}
\usepackage{multirow}

% ----------------------------------------------------
% CONNECTIVES
% ----------------------------------------------------

% Identity
\newcommand{\ax}{\mathsf{ax}}
\newcommand{\cut}{\mathsf{cut}}

% Multiplicatives
\newcommand{\one}{1}
\newcommand{\tensor}{\otimes}
\newcommand{\lolli}{\multimap}

% Additives
\newcommand{\plusl}{\oplus_1}
\newcommand{\plusr}{\oplus_2}
\newcommand{\with}{\&}

% Exponentials
\newcommand{\bang}{{\mathsf{!}}}
\newcommand{\whynot}{{\mathsf{?}}}
\newcommand{\der}{{\mathsf{d}}}
\newcommand{\contr}{{\mathsf{c}}}
\newcommand{\weak}{{\mathsf{w}}}


% ====================================================
% NETS
% ====================================================

\newcommandx{\inet}[4][1=, 4=16]{
	% 1 = style
	% 2 = name
	% 3 = label
	% 4 = number of ports
	\node[net] [#1] (#2) {#3};
	\foreach \i in {-1, 0,...,#4}
		\coordinate (#2_\i) at ($(#2.south west)!\i/(#4+2)!(#2.south east)$);
}

% ====================================================
% CELLS
% ====================================================

\newcommand{\inetcell}[4][]{
	% 1 = style
	% 2 = node name
	% 3 = node symbol
	% 4 = number of ports
	\node[cell] [#1] (#2) {#3};
	\foreach \i in {0, 1,...,#4}
		\coordinate (#2_\i) at ($(#2.corner 2)!\i/#4!(#2.corner 3)$);
}

% ====================================================
% WIRES
% ====================================================

\newcommand{\wire}[3][]{
	% 1 = style
	% 2 = from
	% 3 = to
	\draw[wire] [#1] (#2) -- (#3);
}

\newcommand{\wirelabel}[5][]{
	% 1 = style
	% 2 = from
	% 3 = to
	% 4 = label
	% 5 = position (left, right, above, below)
	\draw[wire] [#1] (#2) -- (#3) node[midway, #5] {#4};
}

\newcommand{\wirepath}[2][]{
	% 1 = style
	% 2 = path
	\draw[wire] [#1] #2;
}

% ====================================================
% BOXES
% ====================================================

\newcommand{\inetbox}[3][]{
	% 1 = style
	% 2 = name
	% 3 = content
	\node[draw,inner xsep=\boxxpadding,inner ysep=\boxypadding,fit=#3] [#1] (#2) {};
	\node[draw,circle,fill=white] at ($(#2.south west)!0.5!(#2.south east)$) (bang) {$\bang$};
}
