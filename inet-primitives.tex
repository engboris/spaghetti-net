\usepackage{xargs}
\usepackage{tikz}
\usetikzlibrary{decorations.pathreplacing}
\usetikzlibrary{decorations.markings}
\usetikzlibrary{calc}
\usetikzlibrary{backgrounds}
\usetikzlibrary{matrix}
\usetikzlibrary{arrows}
\usetikzlibrary{positioning}
\usetikzlibrary{fit}
\usetikzlibrary{shapes.geometric}

% ====================================================
% LAYERS
% ====================================================

\pgfdeclarelayer{bg}
\pgfsetlayers{bg,main}

% ====================================================
% COLORS
% ====================================================

% -------------- Definitions --------------
\definecolor{isabelline}{rgb}{0.96, 0.94, 0.93}

% -------------- Assignment --------------
\newcommand{\netbgcolor}{isabelline}

% ====================================================
% SIZES
% ====================================================

\newcommand{\netypadding}{7pt}
\newcommand{\netxpadding}{35pt}

% ====================================================
% POSITIONING / DISTANCES
% ====================================================

% -------------- Boxes --------------
\newcommand{\boxxpadding}{20pt}
\newcommand{\boxypadding}{30pt}

% ====================================================
% FONTS
% ====================================================

\newcommand{\largenodefontsize}{\scriptsize}
\newcommand{\netfontsize}{\small}
\newcommand{\nodefontsize}{\normalsize}

% ====================================================
% SHAPES
% ====================================================

\tikzset{
%
% -------------- NODE STYLES --------------
  largenode/.style={font=\largenodefontsize},
  downhead/.style={shape border rotate=-180},
  uphead/.style={shape border rotate=360},
  cell/.style={
    anchor=center,
    inner sep=0.5pt,
    regular polygon,
    regular polygon sides=3,
    draw,
    fill=white,
    downhead,
  },
  proofnode/.style={draw, circle, fill=whiteanchor=center, inner sep=0.5pt},
  nospace/.style={inner sep=0pt, outer sep=0pt},
%
% -------------- EDGE STYLES --------------
  % on each segment style
  on each segment/.style={
    decorate,
    decoration={
      show path construction,
      moveto code={},
      lineto code={
        \path [#1]
        (\tikzinputsegmentfirst) -- (\tikzinputsegmentlast);
      },
      curveto code={
        \path [#1] (\tikzinputsegmentfirst)
        .. controls
        (\tikzinputsegmentsupporta) and (\tikzinputsegmentsupportb)
        ..
        (\tikzinputsegmentlast);
      },
      closepath code={
        \path [#1]
        (\tikzinputsegmentfirst) -- (\tikzinputsegmentlast);
      },
    },
  },
  % mid arrow
  mid arrow/.style={postaction={decorate,decoration={
    markings,
    mark=at position .5 with {\arrow[scale=1.5,>=stealth]{>}}
    }}
  },
  % edge styles
  wire/.style={-, shorten <=0.5pt, shorten >=0.5pt, draw=black, line width=0.1ex},
  directed/.style={postaction={on each segment={mid arrow=black}}},
  doublelink/.style={-, double, draw=black, line width=0.15ex, double distance=0.5mm},
%
% -------------- BOX LINE STYLES --------------
  boxline/.style={draw=black, line width=0.1ex, overlay},
%
% -------------- NETS --------------
  net/.style={draw=black, fill=\netbgcolor, inner ysep=\netypadding, inner xsep=\netxpadding, rectangle, anchor=center, font=\netfontsize},
%
% -------------- UNIVERSALS --------------
  every label/.style={label distance=1pt, font=\nodefontsize, inner sep=1pt},
  every node/.style={font=\nodefontsize}
}

\usepackage{amssymb}
\usepackage{amsmath}
\usepackage{graphicx}
\usepackage{fancybox}
\usepackage{xifthen}
\usepackage{cmll}
\usepackage{stmaryrd}
\usepackage{multirow}

% ====================================================
% CONNECTIVES
% ====================================================

% Duality
\newcommand{\dual}[1]{#1^\bot}

% Identity
\newcommand{\ax}{\mathsf{ax}}
\newcommand{\cut}{\mathsf{cut}}

% Multiplicatives
\newcommand{\one}{1}
\newcommand{\tensor}{\otimes}
\newcommand{\lolli}{\multimap}

% Additives
\newcommand{\plusl}{\oplus_1}
\newcommand{\plusr}{\oplus_2}
\def\with{\&}

% Exponentials
\newcommand{\bang}{{\mathsf{!}}}
\newcommand{\whynot}{{\mathsf{?}}}
\newcommand{\der}{{\mathsf{d}}}
\newcommand{\contr}{{\mathsf{c}}}
\newcommand{\weak}{{\mathsf{w}}}


% ====================================================
% NETS
% ====================================================

\newcommandx{\inet}[4][1=, 4=16]{
	% 1 = style
	% 2 = name
	% 3 = label
	% 4 = number of ports
	\node[net] [#1] (#2) {#3};
	\foreach \i in {-1, 0,...,#4}
		\coordinate (#2_\i) at ($(#2.south west)!\i/(#4+2)!(#2.south east)$);
}

% ====================================================
% CELLS
% ====================================================

\newcommand{\inetcell}[4][]{
	% 1 = style
	% 2 = node name
	% 3 = node symbol
	% 4 = number of ports
	\node[cell] [#1] (#2) {#3};
	\foreach \i in {0, 1,...,#4}
		\coordinate (#2_\i) at ($(#2.corner 2)!\i/#4!(#2.corner 3)$);
}

% ====================================================
% WIRES
% ====================================================

\newcommand{\wire}[3][]{
	% 1 = style
	% 2 = from
	% 3 = to
	\draw[wire] [#1] (#2) -- (#3);
}

\newcommand{\wirelabel}[5][]{
	% 1 = style
	% 2 = from
	% 3 = to
	% 4 = label
	% 5 = position (left, right, above, below)
	\draw[wire] [#1] (#2) -- (#3) node[midway, #5] {#4};
}

\newcommand{\wirepath}[2][]{
	% 1 = style
	% 2 = path
	\draw[wire] [#1] #2;
}

% ====================================================
% BOXES
% ====================================================

\newcommand{\inetbox}[3][]{
	% 1 = style
	% 2 = name
	% 3 = content
	\node[draw,inner xsep=\boxxpadding,inner ysep=\boxypadding,fit=#3] [#1] (#2) {};
	\node[draw,circle,fill=white] at ($(#2.south west)!0.5!(#2.south east)$) (bang) {$\bang$};
}
