\input{style}
\usepackage[utf8]{inputenc}
\usepackage{amssymb}
\usepackage{amsmath}
\usepackage{graphicx}
\usepackage{fancybox}
\usepackage{xifthen}
\usepackage{cmll}
\usepackage{stmaryrd}
\usepackage{multirow}

% ----------------------------------------------------
% CONNECTIVES
% ----------------------------------------------------

% Identity
\newcommand{\ax}{\mathsf{ax}}
\newcommand{\cut}{\mathsf{cut}}

% Multiplicatives
\newcommand{\one}{1}
\newcommand{\tensor}{\otimes}
\newcommand{\lolli}{\multimap}

% Additives
\newcommand{\plusl}{\oplus_1}
\newcommand{\plusr}{\oplus_2}
\newcommand{\with}{\&}

% Exponentials
\newcommand{\bang}{{\mathsf{!}}}
\newcommand{\whynot}{{\mathsf{?}}}
\newcommand{\der}{{\mathsf{d}}}
\newcommand{\contr}{{\mathsf{c}}}
\newcommand{\weak}{{\mathsf{w}}}


% ====================================================
% CELLS
% ====================================================

\newcommand{\inetcell}[3][]{
	% 1 = style
	% 2 = node name
	% 3 = node symbol
	\node[cell] [#1] (#2) {#3};
}

% ====================================================
% WIRES
% ====================================================

\newcommand{\wire}[3][]{
	% 1 = style
	% 2 = from
	% 3 = to
	\draw[wire] [#1] (#2) -- (#3);
}

\newcommand{\wirelabel}[5][]{
	% 1 = style
	% 2 = from
	% 3 = to
	% 4 = label
	% 5 = position (left, right, above, below)
	\draw[wire] [#1] (#2) -- (#3) node[midway, #5] {#4};
}

\newcommand{\wirepath}[2][]{
	% 1 = style
	% 2 = path
	\draw[wire] [#1] #2;
}

% ====================================================
% BOXES
% ====================================================

\newcommand{\stboxlw}{6pt}
\newcommand{\stboxh}{20pt}
\newcommand{\stboxrw}{30pt}
\newcommand{\stboxrwaux}{35pt}
\newcommand{\sepbox}{2pt}
\newcommand{\belowbox}{\stboxh+2*\sepbox+1pt}

\newcommand{\boxnodes}[4]{
	% 1 = link to box
	% 2 = left width
	% 3 = right width
	% 4 = height
	\node at (#1.center) [left=#2, nospace] (#1so) {};
	\node at (#1.center) [right=#3, nospace] (#1se) {};
	\node at (#1se.center) [above=#4, nospace] (#1ne) {};
	\node at (#1ne-|#1so) [nospace] (#1no) {};
}

\newcommand{\boxline}[2]{
	% 1 = box link
	% 2 = line style
	\draw[#2](#1.center) -- (#1se.center) -- (#1ne.center) -- (#1no.center) -- (#1so.center)--(#1.center);
}

\newcommand{\abox}[5]{
	% 1 = link to box
	% 2 = line style
	% 3 = right width
	% 4 = left width
	% 5 = height
	\boxnodes{#1}{#3}{#4}{#5}
	\boxline{#1}{#2line}
}

\newcommand{\boxauxnodes}[3]{
	% 1 = link to box
	% 2 = distance of the right conclusion from SE corner
	% 3 = distance between the two conclusions
	\node at (#1se.center)[left = #2, nospace](#1auxk){};
	\node at (#1auxk.center)[left= #3, nospace](#1aux1){};
}

\newcommand{\boxlabel}[2]{
	% 1 = box link
	% 2 = label
	\node at ($(#1so)!.5!(#1ne)$) {#2};
}
