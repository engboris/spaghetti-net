\usepackage{xargs}
\usepackage{tikz}
\usetikzlibrary{decorations.pathreplacing}
\usetikzlibrary{decorations.markings}
\usetikzlibrary{calc}
\usetikzlibrary{backgrounds}
\usetikzlibrary{matrix}
\usetikzlibrary{arrows}
\usetikzlibrary{positioning}
\usetikzlibrary{fit}
\usetikzlibrary{shapes.geometric}

% ====================================================
% LAYERS
% ====================================================

\pgfdeclarelayer{bg}
\pgfsetlayers{bg,main}

% ====================================================
% COLORS
% ====================================================

% -------------- Definitions --------------
\definecolor{isabelline}{rgb}{0.96, 0.94, 0.93}

% -------------- Assignment --------------
\newcommand{\netbgcolor}{isabelline}

% ====================================================
% SIZES
% ====================================================

\newcommand{\netypadding}{7pt}
\newcommand{\netxpadding}{35pt}

% ====================================================
% POSITIONING / DISTANCES
% ====================================================

% -------------- Boxes --------------
\newcommand{\boxxpadding}{20pt}
\newcommand{\boxypadding}{30pt}

% ====================================================
% FONTS
% ====================================================

\newcommand{\largenodefontsize}{\scriptsize}
\newcommand{\netfontsize}{\small}
\newcommand{\nodefontsize}{\normalsize}

% ====================================================
% SHAPES
% ====================================================

\tikzset{
%
% -------------- NODE STYLES --------------
  largenode/.style={font=\largenodefontsize},
  downhead/.style={shape border rotate=-180},
  uphead/.style={shape border rotate=360},
  cell/.style={
    anchor=center,
    inner sep=0.5pt,
    regular polygon,
    regular polygon sides=3,
    draw,
    fill=white,
    downhead,
  },
  proofnode/.style={draw, circle, fill=whiteanchor=center, inner sep=0.5pt},
  nospace/.style={inner sep=0pt, outer sep=0pt},
%
% -------------- EDGE STYLES --------------
  % on each segment style
  on each segment/.style={
    decorate,
    decoration={
      show path construction,
      moveto code={},
      lineto code={
        \path [#1]
        (\tikzinputsegmentfirst) -- (\tikzinputsegmentlast);
      },
      curveto code={
        \path [#1] (\tikzinputsegmentfirst)
        .. controls
        (\tikzinputsegmentsupporta) and (\tikzinputsegmentsupportb)
        ..
        (\tikzinputsegmentlast);
      },
      closepath code={
        \path [#1]
        (\tikzinputsegmentfirst) -- (\tikzinputsegmentlast);
      },
    },
  },
  % mid arrow
  mid arrow/.style={postaction={decorate,decoration={
    markings,
    mark=at position .5 with {\arrow[scale=1.5,>=stealth]{>}}
    }}
  },
  % edge styles
  wire/.style={-, shorten <=0.5pt, shorten >=0.5pt, draw=black, line width=0.1ex},
  directed/.style={postaction={on each segment={mid arrow=black}}},
  doublelink/.style={-, double, draw=black, line width=0.15ex, double distance=0.5mm},
%
% -------------- BOX LINE STYLES --------------
  boxline/.style={draw=black, line width=0.1ex, overlay},
%
% -------------- NETS --------------
  net/.style={draw=black, fill=\netbgcolor, inner ysep=\netypadding, inner xsep=\netxpadding, rectangle, anchor=center, font=\netfontsize},
%
% -------------- UNIVERSALS --------------
  every label/.style={label distance=1pt, font=\nodefontsize, inner sep=1pt},
  every node/.style={font=\nodefontsize}
}

\usepackage{amssymb}
\usepackage{amsmath}
\usepackage{graphicx}
\usepackage{fancybox}
\usepackage{xifthen}
\usepackage{cmll}
\usepackage{stmaryrd}
\usepackage{multirow}

% ====================================================
% CONNECTIVES
% ====================================================

% Duality
\newcommand{\dual}[1]{#1^\bot}

% Identity
\newcommand{\ax}{\mathsf{ax}}
\newcommand{\cut}{\mathsf{cut}}

% Multiplicatives
\newcommand{\one}{1}
\newcommand{\tensor}{\otimes}
\newcommand{\lolli}{\multimap}

% Additives
\newcommand{\plusl}{\oplus_1}
\newcommand{\plusr}{\oplus_2}
\def\with{\&}

% Exponentials
\newcommand{\bang}{{\mathsf{!}}}
\newcommand{\whynot}{{\mathsf{?}}}
\newcommand{\der}{{\mathsf{d}}}
\newcommand{\contr}{{\mathsf{c}}}
\newcommand{\weak}{{\mathsf{w}}}


% ====================================================
% CELLS
% ====================================================

\newcommand{\inetcell}[3][]{
	% 1 = style
	% 2 = node name
	% 3 = node symbol
	\node[cell] [#1] (#2) {#3};
}

% ====================================================
% WIRES
% ====================================================

\newcommand{\wire}[3][]{
	% 1 = style
	% 2 = from
	% 3 = to
	\draw[wire] [#1] (#2) -- (#3);
}

\newcommand{\wirelabel}[5][]{
	% 1 = style
	% 2 = from
	% 3 = to
	% 4 = label
	% 5 = position (left, right, above, below)
	\draw[wire] [#1] (#2) -- (#3) node[midway, #5] {#4};
}

\newcommand{\wirepath}[2][]{
	% 1 = style
	% 2 = path
	\draw[wire] [#1] #2;
}

% ====================================================
% BOXES
% ====================================================

\newcommand{\stboxlw}{6pt}
\newcommand{\stboxh}{20pt}
\newcommand{\stboxrw}{30pt}
\newcommand{\stboxrwaux}{35pt}
\newcommand{\sepbox}{2pt}
\newcommand{\belowbox}{\stboxh+2*\sepbox+1pt}

\newcommand{\boxnodes}[4]{
	% 1 = link to box
	% 2 = left width
	% 3 = right width
	% 4 = height
	\node at (#1.center) [left=#2, nospace] (#1so) {};
	\node at (#1.center) [right=#3, nospace] (#1se) {};
	\node at (#1se.center) [above=#4, nospace] (#1ne) {};
	\node at (#1ne-|#1so) [nospace] (#1no) {};
}

\newcommand{\boxline}[2]{
	% 1 = box link
	% 2 = line style
	\draw[#2](#1.center) -- (#1se.center) -- (#1ne.center) -- (#1no.center) -- (#1so.center)--(#1.center);
}

\newcommand{\abox}[5]{
	% 1 = link to box
	% 2 = line style
	% 3 = right width
	% 4 = left width
	% 5 = height
	\boxnodes{#1}{#3}{#4}{#5}
	\boxline{#1}{#2line}
}

\newcommand{\boxauxnodes}[3]{
	% 1 = link to box
	% 2 = distance of the right conclusion from SE corner
	% 3 = distance between the two conclusions
	\node at (#1se.center)[left = #2, nospace](#1auxk){};
	\node at (#1auxk.center)[left= #3, nospace](#1aux1){};
}

\newcommand{\boxlabel}[2]{
	% 1 = box link
	% 2 = label
	\node at ($(#1so)!.5!(#1ne)$) {#2};
}
