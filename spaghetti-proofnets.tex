% ====================================================
% GENERAL NETS
% ====================================================

% context wire
\newcommand{\context}[4][]{
	% 1 = style
	% 2 = net of reference
	% 3 = node name
	% 4 = label
	\begin{pgfonlayer}{bg}
		\node [#1] (#3) {#4};
		\draw[doublelink] (#2) -- (#3);
	\end{pgfonlayer}
}

% straight line context wire
\newcommand{\straightcontext}[4][]{
	% 1 = style
	% 2 = net of reference
	% 3 = node name
	% 4 = label
	\begin{pgfonlayer}{bg}
		\node[below=5mm of #2] [#1] (#3) {#4};
		\draw[doublelink] (#2) -- (#3);
	\end{pgfonlayer}
}

% ====================================================
% MULTIPLICATIVE CELLS
% ====================================================

\newcommand{\tensorCell}[2][]{
	% 1 = style
	% 2 = node name
	\inetcell[#1]{#2}{$\tensor$}{4}
}

\newcommand{\parCell}[2][]{
	% 1 = style
	% 2 = node name
	\inetcell[#1]{#2}{$\parr$}{4}
}

\newcommand{\oneCell}[2][]{
	% 1 = style
	% 2 = node name
	\inetcell[#1]{#2}{$\parr$}{3}
}

\newcommand{\bottomCell}[2][]{
	% 1 = style
	% 2 = node name
	\inetcell[#1]{#2}{$\bot$}{3}
}

% ====================================================
% ADDITIVE CELLS
% ====================================================

\newcommand{\pluslCell}[2][]{
	% 1 = style
	% 2 = node name
	\inetcell[largenode,#1]{#2}{$\plusl$}{3}
}

\newcommand{\plusrCell}[2][]{
	% 1 = style
	% 2 = node name
	\inetcell[largenode,#1]{#2}{$\plusr$}{3}
}

% ====================================================
% EXPONENTIALS
% ====================================================

\newcommand{\whynotCell}[2][]{
	% 1 = style
	% 2 = node name
	\inetcell[#1]{#2}{$\whynot$}{3}
}

\newcommand{\dereCell}[2][]{
	% 1 = style
	% 2 = node name
	\inetcell[#1]{#2}{$d$}{3}
}

\newcommand{\weakCell}[2][]{
	% 1 = style
	% 2 = node name
	\inetcell[#1]{#2}{$w$}{3}
}

\newcommand{\contrCell}[2][]{
	% 1 = style
	% 2 = node name
	\inetcell[#1]{#2}{$c$}{3}
}

% ====================================================
% BOXES
% ====================================================

% general box without symbol
\newcommand{\inetbox}[3][]{
	% 1 = style
	% 2 = name
	% 3 = content
	\node[draw,inner xsep=\boxxpadding,inner ysep=\boxypadding,fit=#3] [#1] (#2) {};
}

% general box with symbol
\newcommand{\inetboxsym}[4][]{
	% 1 = style
	% 2 = name
	% 3 = content
	% 4 = symbol
	\inetbox[#1]{#2}{#3}
	\node[draw,circle,fill=white] at ($(#2.south west)!0.5!(#2.south east)$) (#2_sym) {#4};
}

\newcommand{\promobox}[3][]{
	% 1 = style
	% 2 = name
	% 3 = content
	\inetboxsym[#1]{#2}{#3}{$\ofcourse$}
}

\newcommand{\withbox}[3][]{
	% 1 = style
	% 2 = name
	% 3 = content
	\inetboxsym[#1]{#2}{#3}{$\with$};
}

% exponential commutation (auxiliary port)
\newcommand{\expcomm}[2][]{
	% 1 = style
	% 2 = node name
	\node[draw,circle,fill=white] [#1] (#2) {$\whynot$};
}

% additive commutation (auxiliary port)
\newcommand{\addcomm}[2][]{
	% 1 = style
	% 2 = node name
	\node[draw,circle,fill=white,largenode] [#1] (#2) {$\whynot_+$};
}
